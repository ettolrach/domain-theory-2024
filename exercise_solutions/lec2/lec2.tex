\chapter{Recursively Defined Programs}

\section{Exercises}

\begin{enumerate}
    \item Give an example of a poset $A$ and a monotonic function $f \colon A \to A$ such that $f$ \textit{doesn't} a fixed point.

    \item What is the lub operator on subsets $X \subseteq \nat$ of the poset $(\nat, \leq)$?

    \item Show that if a function $f \colon A \to B$ on cpos $A$ and $B$ is monotonic and $A$ is finite, then $f$ is continuous.

    Hint: Finite directed sets contain their lub

    \item Show that if a function $f \colon A \to B$ on cpos $A$ and $B$ is monotonic, then $\bigsqcup\{f(x) \mid x \in X\}$ exists.

    Hint: It suffices to show that $\{f(x) \mid x \in X\}$ is directed.
\end{enumerate}

\section{Solutions}

\begin{enumerate}
    \item Take the poset $(\nat, \leq)$ and the function $f \colon \nat \to \nat$ defined to be $f(n) = n+1$.
    Since $f$ doesn't approach any particular value, and since we don't have $\infty$, $\mathbf{fix}(f)$ doesn't exist.
    
    \item We require a binary operator $\sqcup$ such that it is idempotent, symmetric, and associative, and $x \leq y$ iff $x \sqcup y = y$.
    The function $\max \colon \nat \times \nat \to \nat$ defined in the usual way satisfies these requirements.
    For all $n, m, k \in \nat$ where $n \leq m \leq k$,
    \begin{enumerate}[label=\roman*.]
        \item it's idempotent, since $\max(n, n) = n$;
        \item it's symmetric, since $\max(n, m) = m = \max(m, n)$;
        \item it's associative, since
        \begin{equation*}
            \max(n, \max(m, k)) = \max(n, k) = k,
        \end{equation*}
        and
        \begin{equation*}
            \max(\max(n, m), k) = \max(m, k) = k;
        \end{equation*}
        \item and finally, $x \leq y$ iff $x \sqcup y = y$ by the definition of the $\max$ function.
    \end{enumerate}

    \item By the definition of continuity, it only remains to prove that for all directed subsets $X \subseteq A$, it holds that $f(\bigsqcup X) = \bigsqcup\{f(x) \mid x \in X\}$.
    
    Recall that $f$ being monotonic means that for all $a, b \in A$ such that $a \sqsubseteq b$, it holds that $f(a) \sqsubseteq f(b)$.

    Since any subset $X$ of $A$ will be finite, because $A$ is finite, $\bigsqcup X$ will be a single $x_0 \in X$ such that, for all other $x \in X$, $x \sqsubseteq x_0$ (since $X$ is a directed set).
    Because $f$ is monotonic, $f(\bigsqcup X) = f(x_0) \sqsupseteq f(x)$ for all other $x \in X$.

    The image of $f$ under $X$, represented by $\{f(x) \mid x \in X\}$, will have a single member which is ``bigger'' (via the $\sqsubseteq$ relation) than any other; this element is $f(x_0)$, because $f$ is monotonic.
    Thus, $\bigsqcup\{f(x) \mid x \in X\} = f(x_0) = f(\bigsqcup X)$.
    $\square$

    \item Consider some $a, b \in \{f(x) \mid x \in X\}$.
    Let $a_X$ and $b_X$ be the elements which $f$ has mapped $a$ and $b$ from, so $f(a_X) = a$ and $f(b_X) = b$.
    There exists a $c_X \in X$ such that $a_X \sqsubseteq c_X$ and $b_X \sqsubseteq c_X$, since $X$ is a directed set.
    Let $c \in \{f(x) \mid x \in X\}$ be defined as $f(c_X) = c$. 
    Because $f$ is monotonic, $f(a_X) = a \sqsubseteq c = f(c_X)$, and similarly $f(b_X) = b \sqsubseteq c = f(c_X)$. Thus, for any choice of $a, b \in \{f(x) \mid x \in X\}$, one can find a $c$ such that $a \sqsubseteq c$ and $b \sqsubseteq c$, and therefore $\{f(x) \mid x \in X\}$ is directed.
    $\square$
    
\end{enumerate}
